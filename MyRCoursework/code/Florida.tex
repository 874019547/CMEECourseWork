% Options for packages loaded elsewhere
\PassOptionsToPackage{unicode}{hyperref}
\PassOptionsToPackage{hyphens}{url}
%
\documentclass[
]{article}
\usepackage{amsmath,amssymb}
\usepackage{iftex}
\ifPDFTeX
  \usepackage[T1]{fontenc}
  \usepackage[utf8]{inputenc}
  \usepackage{textcomp} % provide euro and other symbols
\else % if luatex or xetex
  \usepackage{unicode-math} % this also loads fontspec
  \defaultfontfeatures{Scale=MatchLowercase}
  \defaultfontfeatures[\rmfamily]{Ligatures=TeX,Scale=1}
\fi
\usepackage{lmodern}
\ifPDFTeX\else
  % xetex/luatex font selection
\fi
% Use upquote if available, for straight quotes in verbatim environments
\IfFileExists{upquote.sty}{\usepackage{upquote}}{}
\IfFileExists{microtype.sty}{% use microtype if available
  \usepackage[]{microtype}
  \UseMicrotypeSet[protrusion]{basicmath} % disable protrusion for tt fonts
}{}
\makeatletter
\@ifundefined{KOMAClassName}{% if non-KOMA class
  \IfFileExists{parskip.sty}{%
    \usepackage{parskip}
  }{% else
    \setlength{\parindent}{0pt}
    \setlength{\parskip}{6pt plus 2pt minus 1pt}}
}{% if KOMA class
  \KOMAoptions{parskip=half}}
\makeatother
\usepackage{xcolor}
\usepackage[margin=1in]{geometry}
\usepackage{color}
\usepackage{fancyvrb}
\newcommand{\VerbBar}{|}
\newcommand{\VERB}{\Verb[commandchars=\\\{\}]}
\DefineVerbatimEnvironment{Highlighting}{Verbatim}{commandchars=\\\{\}}
% Add ',fontsize=\small' for more characters per line
\usepackage{framed}
\definecolor{shadecolor}{RGB}{248,248,248}
\newenvironment{Shaded}{\begin{snugshade}}{\end{snugshade}}
\newcommand{\AlertTok}[1]{\textcolor[rgb]{0.94,0.16,0.16}{#1}}
\newcommand{\AnnotationTok}[1]{\textcolor[rgb]{0.56,0.35,0.01}{\textbf{\textit{#1}}}}
\newcommand{\AttributeTok}[1]{\textcolor[rgb]{0.13,0.29,0.53}{#1}}
\newcommand{\BaseNTok}[1]{\textcolor[rgb]{0.00,0.00,0.81}{#1}}
\newcommand{\BuiltInTok}[1]{#1}
\newcommand{\CharTok}[1]{\textcolor[rgb]{0.31,0.60,0.02}{#1}}
\newcommand{\CommentTok}[1]{\textcolor[rgb]{0.56,0.35,0.01}{\textit{#1}}}
\newcommand{\CommentVarTok}[1]{\textcolor[rgb]{0.56,0.35,0.01}{\textbf{\textit{#1}}}}
\newcommand{\ConstantTok}[1]{\textcolor[rgb]{0.56,0.35,0.01}{#1}}
\newcommand{\ControlFlowTok}[1]{\textcolor[rgb]{0.13,0.29,0.53}{\textbf{#1}}}
\newcommand{\DataTypeTok}[1]{\textcolor[rgb]{0.13,0.29,0.53}{#1}}
\newcommand{\DecValTok}[1]{\textcolor[rgb]{0.00,0.00,0.81}{#1}}
\newcommand{\DocumentationTok}[1]{\textcolor[rgb]{0.56,0.35,0.01}{\textbf{\textit{#1}}}}
\newcommand{\ErrorTok}[1]{\textcolor[rgb]{0.64,0.00,0.00}{\textbf{#1}}}
\newcommand{\ExtensionTok}[1]{#1}
\newcommand{\FloatTok}[1]{\textcolor[rgb]{0.00,0.00,0.81}{#1}}
\newcommand{\FunctionTok}[1]{\textcolor[rgb]{0.13,0.29,0.53}{\textbf{#1}}}
\newcommand{\ImportTok}[1]{#1}
\newcommand{\InformationTok}[1]{\textcolor[rgb]{0.56,0.35,0.01}{\textbf{\textit{#1}}}}
\newcommand{\KeywordTok}[1]{\textcolor[rgb]{0.13,0.29,0.53}{\textbf{#1}}}
\newcommand{\NormalTok}[1]{#1}
\newcommand{\OperatorTok}[1]{\textcolor[rgb]{0.81,0.36,0.00}{\textbf{#1}}}
\newcommand{\OtherTok}[1]{\textcolor[rgb]{0.56,0.35,0.01}{#1}}
\newcommand{\PreprocessorTok}[1]{\textcolor[rgb]{0.56,0.35,0.01}{\textit{#1}}}
\newcommand{\RegionMarkerTok}[1]{#1}
\newcommand{\SpecialCharTok}[1]{\textcolor[rgb]{0.81,0.36,0.00}{\textbf{#1}}}
\newcommand{\SpecialStringTok}[1]{\textcolor[rgb]{0.31,0.60,0.02}{#1}}
\newcommand{\StringTok}[1]{\textcolor[rgb]{0.31,0.60,0.02}{#1}}
\newcommand{\VariableTok}[1]{\textcolor[rgb]{0.00,0.00,0.00}{#1}}
\newcommand{\VerbatimStringTok}[1]{\textcolor[rgb]{0.31,0.60,0.02}{#1}}
\newcommand{\WarningTok}[1]{\textcolor[rgb]{0.56,0.35,0.01}{\textbf{\textit{#1}}}}
\usepackage{graphicx}
\makeatletter
\def\maxwidth{\ifdim\Gin@nat@width>\linewidth\linewidth\else\Gin@nat@width\fi}
\def\maxheight{\ifdim\Gin@nat@height>\textheight\textheight\else\Gin@nat@height\fi}
\makeatother
% Scale images if necessary, so that they will not overflow the page
% margins by default, and it is still possible to overwrite the defaults
% using explicit options in \includegraphics[width, height, ...]{}
\setkeys{Gin}{width=\maxwidth,height=\maxheight,keepaspectratio}
% Set default figure placement to htbp
\makeatletter
\def\fps@figure{htbp}
\makeatother
\setlength{\emergencystretch}{3em} % prevent overfull lines
\providecommand{\tightlist}{%
  \setlength{\itemsep}{0pt}\setlength{\parskip}{0pt}}
\setcounter{secnumdepth}{-\maxdimen} % remove section numbering
\ifLuaTeX
  \usepackage{selnolig}  % disable illegal ligatures
\fi
\IfFileExists{bookmark.sty}{\usepackage{bookmark}}{\usepackage{hyperref}}
\IfFileExists{xurl.sty}{\usepackage{xurl}}{} % add URL line breaks if available
\urlstyle{same}
\hypersetup{
  pdftitle={Is Florida getting warmer},
  hidelinks,
  pdfcreator={LaTeX via pandoc}}

\title{Is Florida getting warmer}
\author{}
\date{\vspace{-2.5em}}

\begin{document}
\maketitle

\hypertarget{is-florida-getting-warmer}{%
\subsection{1.Is Florida getting
warmer?}\label{is-florida-getting-warmer}}

\hypertarget{section}{%
\subsubsection{1)}\label{section}}

\begin{Shaded}
\begin{Highlighting}[]
\FunctionTok{load}\NormalTok{(}\StringTok{"../data/KeyWestAnnualMeanTemperature.RData"}\NormalTok{)}
\NormalTok{mycor }\OtherTok{\textless{}{-}} \FunctionTok{cor}\NormalTok{(ats}\SpecialCharTok{$}\NormalTok{Year, ats}\SpecialCharTok{$}\NormalTok{Temp)}
\NormalTok{mycor}
\end{Highlighting}
\end{Shaded}

\begin{verbatim}
## [1] 0.5331784
\end{verbatim}

\hypertarget{section-1}{%
\subsubsection{2)}\label{section-1}}

\begin{Shaded}
\begin{Highlighting}[]
\FunctionTok{set.seed}\NormalTok{(}\DecValTok{1234}\NormalTok{)}
\NormalTok{num\_permutations }\OtherTok{\textless{}{-}} \DecValTok{10000}
\NormalTok{permuted\_corrs }\OtherTok{\textless{}{-}} \FunctionTok{numeric}\NormalTok{(num\_permutations)}
\ControlFlowTok{for}\NormalTok{ (i }\ControlFlowTok{in} \DecValTok{1}\SpecialCharTok{:}\NormalTok{num\_permutations) \{}
\NormalTok{  shuffled\_temps }\OtherTok{\textless{}{-}} \FunctionTok{sample}\NormalTok{(ats}\SpecialCharTok{$}\NormalTok{Temp)}
\NormalTok{  permuted\_corr }\OtherTok{\textless{}{-}} \FunctionTok{cor}\NormalTok{(ats}\SpecialCharTok{$}\NormalTok{Year, shuffled\_temps)}
\NormalTok{  permuted\_corrs[i] }\OtherTok{\textless{}{-}}\NormalTok{ permuted\_corr}
\NormalTok{\}}
\end{Highlighting}
\end{Shaded}

\hypertarget{section-2}{%
\subsubsection{3)}\label{section-2}}

\begin{Shaded}
\begin{Highlighting}[]
\NormalTok{p\_value }\OtherTok{\textless{}{-}} \FunctionTok{sum}\NormalTok{(}\FunctionTok{abs}\NormalTok{(permuted\_corrs) }\SpecialCharTok{\textgreater{}=} \FunctionTok{abs}\NormalTok{(mycor))}\SpecialCharTok{/}\NormalTok{num\_permutations}
\end{Highlighting}
\end{Shaded}

\hypertarget{section-3}{%
\subsubsection{4)}\label{section-3}}

\hypertarget{goal}{%
\subsection{Goal}\label{goal}}

The goal of this analysis was to investigate the correlation between
years and temperatures in Florida. We used permutation analysis to
assess the significance of the observed correlation coefficient.

\hypertarget{results}{%
\subsection{Results}\label{results}}

\begin{itemize}
\tightlist
\item
  \textbf{Observed Correlation Coefficient:} 0.5331784
\item
  \textbf{Permutation p-value:} 0
\end{itemize}

\hypertarget{interpretation}{%
\subsection{Interpretation}\label{interpretation}}

The observed correlation coefficient(0.5331784) represents the
relationship between years and temperatures in Florida.

The calculated p-value from the permutation test is 0, it suggests that
the observed correlation coefficient (r=0.5331784) is statistically
significant and highly unlikely to occur by random chance alone. This
means there is strong evidence to support the claim that Florida is
getting warmer over the years. The p-value of 0 indicates that the
observed correlation coefficient falls far outside the range of
correlation coefficients obtained by randomly reshuffling the
temperatures, implying a significant and positive correlation between
years and temperatures in Florida.

\end{document}
